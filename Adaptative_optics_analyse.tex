\documentclass{article}
\usepackage[french]{babel}
\usepackage{graphicx}
\usepackage[utf8]{inputenc}
\usepackage[T1]{fontenc}

\begin{document}


\title{Analysis of multiconjugate adaptative optics : analyse de papier scientifique }
\author{Gonnier Noémie, Darpentigny Nathan, Chalot Tanguy}

\maketitle

\section{Introduction}
Présentation du document 

\section{Enjeux}

Cette section consiste en un résumé de l'article
\subsection{Contexte et enjeux (?)}

Un mot sur le contexte et l'état de l'art ( vaguement, si on trouve des trucs) en 1992, et présenter les enjeux du papier. Présenter le laboratoire ( Department of Electrical and Computer Engineering US Air Force institute of Technology) et son domaine de recherche.//
Présenter ensuite les enjeux du documents et comment il se place. //
\subsection{Rappels sur l'optique adaptative}
Rappeler le principe générale de l'optique adaptative. ( enfin, les principes dont on a besoin ) . C'est ke rapprochement avec le cours. 

\subsection{Principe du travail}

Dans cette section on résume l'article : on va faire des dessins pour qu'on comprenne bien ce sur quoi porte l'analyse et définir toutes les variables utilisées dans les résultats.

\subsection{Résultats marquants}

Ecriture des résultats principaux et explication physique, que veulent ils dire. 

\section{Originalité}
Dans cette section, discussion autour de l'originalité et pertinence du résultat : //

Par rapport aux travaux précédent du lab//
Par rapport aux publications de l'époque//
Et surtout dans les documents qui le citent et l'utilisent. 



\end{document}